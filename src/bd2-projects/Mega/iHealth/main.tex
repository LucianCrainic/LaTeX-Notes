\documentclass{homework}

\title{iHealth}
\author{Basi di Dati 2}
\date{}

\begin{document}

\maketitle
Si vuole progettare iHealth, un sistema web che permetta agli utenti di 
prenotare   visite   mediche   specialistiche,   anche   scegliendo   i   
mediciaderenti   al   circuito.   I   medici   possono   appartenere   sia  
 a   strutture pubbliche che private.iHealth deve consentire agli utenti, previa registrazione e autenticazione,di consultare l'elenco dei medici aderenti al circuito, potendo filtrare lalista per città, specializzazione clinica (ad es., "oftalmologia"), tipologia divisita   (relativa   ad   una   specializzazione,   ad   es.,   "visita   oculisticastandard"), e tipologia della struttura dove operano (pubblica o privata).Ogni medico che vuole rendersi disponibile attraverso il circuito iHealthdeve registrarsi utilizzando i propri dati anagrafici: nome, cognome, datadi nascita, codice fiscale, email, matricole dei Sistemi Sanitari Nazionalidei paesi dove è abilitato a operare, città e strutture dove intendeoperare e specializzazioni mediche per le quali è abilitato (delle quali unaè la specializzazione primaria). Sebbene un medico possa operare indiverse città e presso diverse strutture, il sistema deve permettere aimedici di dichiararsi operativi solo nei paesi nei quali sono legalmenteabilitati ad operare dai rispettivi Sistemi Sanitari Nazionali.Durante la registrazione, anche gli utenti devono dichiarare i propri datianagrafici (nome, cognome, data di nascita, codice fiscale, email, città diresidenza e, opzionalmente, città di domicilio abituale). Tuttavia, mentrela registrazione degli utenti è immediata, quella dei medici deve essereapprovata dal management di iHealth, che verifica la veridicità dei datiinseriti. Prima di tale approvazione, un medico neo-registrato non puòessere oggetto di prenotazione per visite.Anche le strutture sanitarie pubbliche e private nelle quali i medici diiHealth   operano   (ad   es.,   ospedali   o   cliniche)   sono   soggette   aregistrazione   e   verifica   preventiva.   In   particolare,   un   medico   puòdichiarare di operare in una certa struttura solo se quest'ultima è statagià registrata nel circuito dal relativo management, che avrà fornito:nome della struttura, Partita IVA o Codice Fiscale della struttura (il CodiceFiscale va indicato in caso di assenza di Partita IVA, ad es., per lestrutture sanitarie no-profit), indirizzi delle diversi sedi operative, enome, cognome, data di nascita, codice fiscale ed email del responsabilelegale della struttura. Inoltre, durante la registrazione di una struttura,vanno anche indicate le specializzazioni cliniche e le tipologie di visitaofferte nelle diversi sedi operative. Per ogni sede operativa e per ognitipologia di visita di ogni specializzazione offerta in quella sede, va
indicato se le visite di tale tipologia vengono offerte in tale sede inconvenzione con il sistema sanitario nazionale del paese nel quale lasede operativa è collocato e/o interamente a pagamento.Si noti che la presenza di una anagrafica completa per le strutturesanitarie implica che un medico non può offrire visite per una certaspecializzazione clinica (anche se in suo possesso) in una certa sede diuna certa struttura, se tale specializzazione non è dichiarata offerta intale sede.iHealth, oltre a permettere agli utenti, come anticipato, la ricerca dimedici filtrata per città, specializzazione clinica, tipologia di visita, etipologia di struttura, deve consentire la prenotazione di visite.All'atto   della   prenotazione   di   una   visita,   l'utente   deve   scegliere:   ilmedico, una sede (che è una delle sedi operative di una struttura dove ilmedico opera), e dichiarare se la visita è da effettuarsi in convenzionecon il Servizio Sanitario Nazionale del paese dove la sede della strutturaopera   o   interamente   a   pagamento.   All'atto   della   richiesta   di   unaprenotazione,   iHealth   proporrà   all'utente   una   lista   di   date/ore   perl'appuntamento, tra le quali l'utente potrà scegliere, concludendo quindiil processo.Le date/ore per gli appuntamenti vengono proposte automaticamente daiHealth   nel   modo   seguente.   Ogni   medico   dichiara,   per   ogni   sedeoperativa dove opera e per ogni tipologia di visita che intende offrire, ladurata   attesa   della   visita   standard.   Inoltre,   il   medico   dichiaraesplicitamente un insieme di giorni e fasce orarie nei quali è operativo inquella sede (iHealth eviterà sovrapposizioni dei periodi di disponibilitàdello stesso medico in sedi diverse). Il sistema quindi proporrà agli utenti,come   data/ora   dell'appuntamento   relativo   alla   visita   in   corso   diprenotazione, tutte le date/ore compatibili con le disponibilità dichiaratedel medico e con la durata attesa della visita, e non in conflitto con visitegià prenotate, e che non lascino buchi di tempo inutilizzato nell'agendadel medico.Il sistema dovrà infine permettere agli utenti di cancellare una visita giàprenotata. La cancellazione può ovviamente avvenire solo prima delmomento nel quale la visita è prevista.

\end{document}