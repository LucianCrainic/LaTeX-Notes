\documentclass{homework}

\title{The mathematical justice machine}
\author{Lucian D. Crainic}

\begin{document}

\maketitle
The paper describes the functioning of an administrative system (the Machine) 
which seems to guarantee absolute justice, eliminating any form of favoritism or 
dishonesty. \\
The author appreciates the efficiency of the drilling machine, which 
has just secured his promotion, and recognizes its usefulness in the Administration.
However, the author admits that he feels disappointed 
because, despite the efficiency of the machine, he hasn't gotten what he really 
wanted: to be loved. This fact made him discover that what he wanted was not only
a promotion, but also personal recognition. \\
The author seems to express a criticism of technology and its influence on human 
life. Also he seems to suggest that the Machine described is a metaphor for 
humanity itself, and that its power has become so great as to make man a passive 
object of control. The Machine is portrayed as perfect, but also incapable of 
emotion or compassion, and this makes it inhuman. Furthermore, the Machine seems 
to try to shape men in its image, reducing them to mere objects to be manipulated.
The author suggests that man should be free to develop without being subject to 
this kind of control.\\
The author reflects on the limitations of the Machine, which can only respond within 
the limits of what is perforable. While the Machine is always truthful, it lacks 
the ability to deal with complex issues beyond its programming. He questions 
whether the Machine's inability to handle certain issues is due to their complexity 
or simply the limitations of the perforation system.
The author also reflects on the old style of communication, before the advent of 
perforation, where it was possible to include personal considerations in official 
letters. While this was not considered proper, it allowed for a more fluid exchange 
of ideas and information.\\
The author seems to question the idea that the Machine is capable of solving any 
problem and suggests that man has an important role in maintaining a balance between 
the perfection of the Machine and the human ability to make mistakes and to learn from them own mistakes. 
Furthermore, he seems to be concerned that the use of the Machine may 
lead to the loss of man's ability and creativity. \\
The text argues that the machine has not freed man, even though it has 
deprived him of every decision and the opinions of individuals have become rare. 
The engineers who build the machine think they understand it, but in reality 
they are lost in their own reflection. The author notes that people are no longer heard to 
say "if I were the machine", perhaps because they no longer dare to say "if I were smarter
or more capable".  
Finally, he argues that since the machine is incapacitated by construction, it has 
the right to contradict itself. \\
The author expresses perplexity towards the machine, which is 
endowed with an unprecedented perfection but also with an incredible blindness. 
Although the machine is perfect and does not make mistakes, it runs into serious 
failures. According to the official theses of the specialists, everything is 
foreseeable provided that sufficient data are available, but the narrator cannot 
be satisfied with such theories. The degree of perfection of the machine is 
measured on the basis of its complication, which must be sufficient to be 
equivalent to reality. The failures of the machine consist in the fact that 
whenever it plays on the administrative chessboard, reality always responds 
with an ironic backlash. He is perplexed by the need to complicate 
the machine more and more to perfect it. \\
The author argues that machines are neither good nor bad in themselves, 
and can be useful when used correctly. However, the author is critical of the fact 
that machines have taken over some of the functions that were once performed by 
humans, leaving them feeling displaced and uncertain about their own identities. 
The author compares this feeling to the desperation of mystics seeking the 
knowledge of impersonality. Despite this, he acknowledges that machines 
are not simply caricatures of providence, but rather a pure being that provides 
the law that cannot be deceived or violated without consequences. \\
The machine is built to never contradict itself and that if it tried, it 
would break. The author notes that human beings are similar in that they 
cannot contradict each other without negative consequences. However, 
the author notes that human beings can lie, which is different from 
contradiction. 
The author argues that truth and falsehood are both often misused, 
but this is not important because they are both essential to human 
culture. He claims that contradiction is a specifically 
human fact and that all our actions are contradictory in some way.
He argues that our mistakes and our lies reveal us to 
ourselves and that every mask we wear reveals an authentic aspect 
of ourselves that has not yet been expressed. The author concludes 
that philosophy and administration try to force man to force his 
inner contradiction into silence, but this effort has failed and 
the contradiction continues to exist. The machine ignores the 
contradiction, but the police fear it and try to repress any 
contradictory speech.

\end{document}